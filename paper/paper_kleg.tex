\documentclass{comjnl}

\usepackage{amsmath}
\usepackage{amsfonts}

%Copyright 2014 Victor Villas Bôas Chaves

\begin{document}


\title[Kirchhoff's laws as Linear Equations over a RCL Electrical Circuit modelled in Graphs]{Kirchhoff's laws as Linear Equations over a RCL Electrical Circuit modelled in Graphs}
\author{Victor Chaves}
\affiliation{Military Institute of Engineering, Rio de Janeiro, Brazil} \email{victor.slosilojn@gmail.com}

\shortauthors{V. Chaves}

%\received{00 January 2009}
%\revised{00 Month 2009}


%\category{C.2}{Computer Communication Networks}{Computer Networks}
%\category{C.4}{Performance of Systems}{Analytical Models}
%\category{G.3}{Stochastic Processes}{Queueing Systems}
%\terms{Internet Technologies, E-Commerce}
\keywords{Graphs, Linear Algebra, Electricity Circuits}


\begin{abstract}
Graphs and circuits are so closely related that an undergraduate course about electricity networks that does not mention graph modelling should be considered a scientific blasphemy. Here are presented the basic aspects of this relationship, linking the linear behaviour of generalized Ohm's Law and a linear model: Kirchoff's Laws using some residual linear algebra born from graph's incidence matrix. In fact, so little graph theory is required that not knowing graph theory is a poor excuse to ignore this approach. Some simple sample circuits are analysed using a software derived directly from the model discussed and it's source code is made available so that anyone can try their own tests.
\end{abstract}

\maketitle


\section{Introduction}
\label{Sec:Intro}

By it's very definition, electrical networks can be modelled by graphs. The visual similarity may suggest that we model electrical nodes with graph nodes and electrical components within graph edges.

%But a big part of the information required to solve a electrical circuit network problem - that is, given the electrical components, find the currents and voltages on the components - relies on information found on branches, and graphs usually tend to put most of the information on the nodes. It is understandable that many people think first in modelling electrical components within graph nodes and the relation "shares a node on the circuit" with graph edges.

There is no such a thing as "the correct model", only "useful model", "not so useful model" and "probably useless model". In this paper we shall see how useful is the visually intuitive model. No prior knowledge in graphs and liner algebra is required, probably a few minutes of reading can put the reader in touch with every concept in this work.

\section{Let there be Vectors and Laws}
Let $\mathcal{C}$ be an electrical circuit network of $n$ nodes and $m$ branches containing electrical connections, e.g. wires, resistors and batteries. Each node has an electrical potential, and each branch is associated with the potential difference between it's terminals and the current which goes through the electrical component in the branch.

Fixing the current directions along, we may define the currents and voltage vectors:
\begin{definition}[Branch Current Vector]
$$ \vec{i} \in \mathbb{C}^m$$ where $\vec{i}_k$ is the current going through the k-th branch in the prefixed direction.
\end{definition}
\begin{definition}[Branch Voltage Vector]
$$ \vec{v} \in \mathbb{C}^m$$ where $\vec{v}_k$ is the potential drop along the k-th branch in the prefixed direction.
\end{definition}
\begin{definition}[Node Potential Vector]
$$ \vec{p} \in \mathbb{C}^n$$ where $\vec{p}_k$ is the potential on each node.
\end{definition}

\begin{remark}[Voltage as potential differences]
One simple observation is the relation between the voltage vector and the potential vector. If $b=(u,v)$ is a branch and the prefixed direction for this branch is from node $u$ to node $v$, then $\vec{v}_b = \vec{p}_u - \vec{p}_v$.
\end{remark}

\begin{remark}[Uniqueness]
\label{Re:Uniqueness}
Students of electrical networks problems may observe that the solution is not unique: if $\vec{p}$ is a solution, $\vec{p}+\alpha(1,1,...,1)$ is also a solution, for every $\alpha \in \mathbb{C}$. For this reason, it is common to choose a node to "ground", that is, set it's potential to zero. Later we shall discuss multiplicity of solutions under linear algebra theorems.
\end{remark}

The Ohm's Law in the complex form gives the relation between the voltage and the current on each RLC branch:
\begin{axiom}[Ohm]
$$V = ZI $$
\end{axiom}
where $V$ is the voltage of an electrical component, $I$ is the current passing through it and $Z$ is the impedance associated with the component. 

Of course these relations are the model of the circuit properties, but we need behaviour laws so we can predict the properties of a given circuit. These laws can be intuitively deduced from energy conservation principles: nodes do not leak or storage charges and cyclic paths can't generate work.

These assumptions are not always perfectly satisfied by electrical networks, but are in most cases enough to well approximate the behaviour of electrical charges in the system. The reason behind these assumptions can be naively explained by saying that if one of these were not true, the system wouldn't be physically stable.

Finally, we can state the famous Kirchhoff's Laws that together with Ohm's Law describe the electrical system in a totally linear model (although Kirchhoff's may not seem linear at first glance):
\begin{axiom}[Current Law]
The sum of all currents entering a node must be equal to the sum of all currents leaving the node.
\end{axiom}
\begin{axiom}[Voltage Law]
The sum of branch voltages along a cyclic path must be equal to zero.
\end{axiom}

The Kirchhoff's Current Law (KCL) is the formal axiom behind our first assumption and the Kirchhoff's Voltage Law (KVL) is behind the second. They are so intuitive that most people don't stare with proper wonder and awe the linear aspect given by the words "sum of all" or "sum of ... along"

While dealing with linear systems and models we will obviously get support from linear algebra. The mathematical structure related to networks and mappings that is also related to matrices is the graph, and that's the whole purpose of this writing.

\section{Kirchhoff enters the Matrix}
If we choose to represent electrical circuit networks under a set of prefixed current directions, then we are intuitively representing the electrical network as a directed multi-graph. "Directed" because the edges of the graph have a defined direction and "multi" because more than one edge linking two vertices is allowed (useful because electrical components in parallel).

The reader unfamiliar with graphs need not to worry, since graph's only purpose here is to formalize an important auxiliary data structure of networks called incidence matrix.

\begin{definition}[Incidence Matrix]
Let $G(V,E)$ be a graph having $V$ as the set of vertices and $E$ as the set of edges. The incidence matrix $A(G)$ of this graph is given by:
$$ A_{|V|\times|E|} :=
\begin{cases} 
a_{ij}=+1 \mbox{ if edge j arrives at vertice i}\\
a_{ij}=-1 \mbox{ if edge j leaves from vertice i}\\
a_{ij}=0 \mbox{ if edge j don't touch vertice i}\\
\end{cases}$$
\end{definition}

Now, given the directed multi-graph $G(V,E)$ modelling a circuit $\mathcal{C}$, are uniquely defined the incidence matrix $A$, the branch voltage vector $\vec{v}$, the branch current vector $\vec{i}$ and the node potentials vector $\vec{p}$.

With linear algebra approach, we formulate the Kirchhoff's Current Law with matrix operations:
\begin{theorem}[KCL]
\label{KCL:matrix}
$$A\vec{i} = \vec{0}$$
\end{theorem}
\begin{proof}
Considering the algebraic sum of currents (positive currents enter the node and negative currents leave the node), then KCL in the k-th node gives:
$$(A\vec{i})_k = \sum a_{kj}i_j = \sum_{entering} \vec{i}_x - \sum_{leaving} \vec{i}_y = 0 $$
\end{proof}

Besides the mathematical beauty of a simpler KCL, even more interesting is the observation that the Kirchhoff's Voltage Law follows directly from voltage vector definition as potential differences.

\begin{theorem}[KVL]
\label{KVL:matrix}
$$ A^T\vec{p}=-\vec{v} \Rightarrow \mbox{KVL}  $$
\end{theorem}
\begin{proof}
First we prove the linear equation. Let $b=(u,v)$ be the k-th edge, going from $u$ to $v$:
$$(A^T\vec{p})_k = \sum a_{jk}p_j = \vec{p}_u - \vec{p}_v = \vec{v}_b = v_k$$
Now, any cyclic path results in a null linear combination in the column space of $A^T$. Hence, the voltage drop of the path is a null combinations of the node potentials along the path:
$$ \sum_{cyclic}v_k = - \sum_{cyclic} (A^T\vec{p})_k  = - \sum_{cyclic} (A^T)_k \cdot \vec{p} = \vec{0} \cdot \vec{p} = 0$$
\end{proof}

In order to use the Generalized Ohm's Law we must separate the voltage and current sources from the circuit. We can do this by defining the voltage drop vector $\vec{v}_{d}$ and voltage source contribution vector $\vec{v}_{s}$. In a similar fashion we define $\vec{i}_{s}$ as the current sources contribution vector and $\vec{i}_{d}$ so that:
$$ \vec{v} = \vec{v}_{d} - \vec{v}_{s}$$
$$ \vec{i} = \vec{i}_{d} +  \vec{i}_{s}$$

And now we can securely state the Generalized Ohm's Law:
\begin{theorem}[GOL]
\label{GOL:matrix}
$$ \vec{v}_{d} = Z \vec{i}_{d} $$
where $Z$ is the impedance matrix:
$$ Z_{|E|\times|E|} :=
\begin{cases} 
Z_{jj} \mbox{ is the impedance on branch j}\\
Z_{ij}=0 \mbox{, if } i\neq j \\
\end{cases}$$
\end{theorem}
Combining \ref{KCL:matrix}, \ref{KVL:matrix} and \ref{GOL:matrix} we have the complete linear description of the system:
\begin{equation}
\label{Eq:CompleteModel}
\begin{bmatrix}
	Z & A^T \\
	A & 0   \\
\end{bmatrix}
\begin{bmatrix}
	\vec{i}_d \\
	\vec{p}
\end{bmatrix}
=
\begin{bmatrix}
	\vec{v}_s \\
	-A\vec{i}_s
\end{bmatrix}
\end{equation}

The incredulous reader may expand the equation as follows:
$$ Z\vec{i}_d + A^T\vec{p} = \vec{v}_d - \vec{v} = \vec{v}_s $$
$$ A\vec{i}_d = A(\vec{i}-\vec{i}_s) = - A\vec{i}_s  $$

\section{Linear Space: the final frontier}
Before trying to solve problems using equation \ref{Eq:CompleteModel}, we must keep in mind remark \ref{Re:Uniqueness}. The solution is already known: ground a node.

While many may find physically intuitive that the ground node could be chosen with no loss of generality, a serious mathematical modelling paper must prove this claim. 

\begin{theorem}[Rank of A]
$$Rnk(A)=|V|-1$$
\end{theorem}
\begin{proof}
Of course $Rnk(A)<|V|$, because the sum of all rows is the null vector. Now, choose the smallest $0<k<|V|$ such that there are $k$ rows linearly dependent.

Because the graph is connected, there must be an edge $m$ going from the group of nodes represented by the $k$ chosen rows and the $|V|-k>0$ nodes left. Hence, in the $k$ rows chosen, only one has a non-zero element on the $m$-th column. So, this row cannot participate on the null linear combination and we would have $k-1$ linearly dependent rows, contradicting the hypothesis.
\end{proof}


\begin{theorem}[Null Space of A]

\end{theorem}

% Ground a node, and solve the goddamn problem!



\section{Numerical Examples}

% Need to finish implementation, even small examples are just too hard to do by hand

\section{Conclusions}
\label{Conclusions}


\nocite{*}

\bibliographystyle{compj}
\bibliography{ModellingBidders}


\end{document}
