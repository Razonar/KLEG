\documentclass{comjnl}

\usepackage{amsmath}

%\copyrightyear{2014} \vol{00} \issue{0} \DOI{000}

\begin{document}


\title[Kirchhoff's laws for Linear Equations over a circuit modelled in Graphs]{Kirchhoff's laws for Linear Equations over a circuit modelled in Graphs}
\author{Victor Chaves}
\affiliation{Military Institute of Engineering, Rio de Janeiro, Brazil} \email{victor.slosilojn@gmail.com}

\shortauthors{V. Chaves}

%\received{00 January 2009}
%\revised{00 Month 2009}


%\category{C.2}{Computer Communication Networks}{Computer Networks}
%\category{C.4}{Performance of Systems}{Analytical Models}
%\category{G.3}{Stochastic Processes}{Queueing Systems}
%\terms{Internet Technologies, E-Commerce}
\keywords{Graphs, Linear Algebra, Electricity Circuits}


\begin{abstract}
Here will go the abstract
\end{abstract}

\maketitle


\section{Introduction}
\label{Sec:Intro}

By it's very definition, electrical circuits can be well modelled by graphs. The vertices may carry the nodes' voltage information if we set a ground referential and the edges represent the electrical components, so they can carry their impedance.

We shall see how common graph auxiliary data structures can be used to find the currents in each electrical component, mainly using matrix operations. The generalized Ohm's Law gives us a linear equation to be solved involving these matrices.



\subsection{Electrical Circuits and Components}
\label{Sec:Circuits&Components}

\subsection{Kirchhoff's Laws}
\label{Sec:Kirchhoff}

\subsection{Complex Numbers and Phasor Notation}
\label{Sec:Phasors}

\subsection{Generalized Ohm's Law}
\label{Sec:Ohm}

\section{Graph Modeling}
\label{Sec:Graph}


\section{Performance measures of interest to the SB and to the seller}  \label{Performance}


\section{Test}


\section{Price dependent bidding} \label{Price-Dependent}

In many cases the current price attained by a good offers useful
information about its valuation, and about the situation of other
bidders. Thus a model with bidding rates dependent on price was
analysed in \cite{gelenbe06}. Here we extend this approach to the
behaviour of both the SB and the other bidders.

We use $\beta(l)$ and $\lambda(l)$ to denote the bidding rates
when the price is at level $l$ for the SB and the other bidders,
respectively. Likewise, $\delta(l)$ will be the seller's decision
rate when price is at level  $l$. By a simple extension of the
previous model, the steady state probabilities for the system
satisfy
\begin{align} \label{eq:PriceDependentBidding}
P(O(1))&= \frac{n\lambda(0)}{(n-1)\lambda({1}) + \beta(1) + \delta(1) } P(0),  \\
P(R(1))&= \frac{\beta(0)}{n \lambda(1) + \delta(1)} P(0) ,\nonumber \displaybreak[0]\\
P(O(l))&= \frac{(n-1) \lambda({l-1})}{(n-1)\lambda(l) +
\beta(l)+\delta(l)} P(O(l-1)) \nonumber\\
&\quad + \frac{n \lambda({l-1})}{(n-1)\lambda(l) + \beta(l) + \delta(l)} P(R(l-1)) , \,\nonumber\\
&\qquad \qquad \qquad \qquad\qquad\qquad 2\leq l \leq v-1 , \nonumber \displaybreak[0] \\
P(R(l)) &= \frac{\beta(l-1)}{n \lambda(l) + \delta(l)} P(O(l-1)) , \quad 2\leq l \leq v-1 ,\nonumber\displaybreak[0] \\
P(O(l)) &= \frac{(n-1)\lambda(l-1)}{\delta(l)} P(O(l-1)) \nonumber \\
&\quad + \frac{n\lambda(l-1)}{\delta(l)} P(R(l-1)) , \quad  l=v , \nonumber \displaybreak[0]\\
P(R(l)) &= \frac{\beta(l-1)}{\delta(l)} P(O(l-1)) , \quad l=v ,\nonumber \displaybreak[0]\\
P(A(O,l)) &= \frac{\delta(l)}{r} P(O(l)) ,\quad 1\leq l \leq v,
\nonumber \displaybreak[0]\\
P(A(R,l)) &= \frac{\delta(l)}{r} P(R(l)) ,\quad 1\leq l \leq v,
\nonumber \displaybreak[0]\\
P(0) &= \frac{r}{n\lambda(0)+\beta(0)} \sum_{U=O,R}\sum_{l=1}^v P(A(U,l)), \nonumber\\
1 &=P(0)+\sum_{U=0,R}\sum_{l=1}^v \big[P(U(l))+P(A(U,l))\big].
\nonumber
\end{align}

We will first give the general solutions for this system, and then
look at a plausible example of forms that the dependent functions
$\lambda, \beta$ and $\delta$ might assume. Suppose, following
similar approach in (\ref{eq:1_3}), we let
\begin{align*}
P(O(l)) &= H(l) P(0),  \quad 1\leq  l \leq v, \\
P(R(l)) &= G(l) P(0), \quad 1 \leq l \leq v.
\end{align*}
We can then express the second order recurrence relations in
$H(l)$:
\begin{equation}\label{eq:recurr_PriceDep}
H(l) = c_1(l) H(l-1) + c_2(l) H(l-2), \quad 3 \leq l \leq v-1,
\end{equation}
where the coefficients $c_1$ and $c_2$ are
\begin{align}\label{eq:recurr_PriceDep_coeff}
c_1(l) &=
\frac{(n-1)\lambda(l-1)}{(n-1)\lambda(l)+\beta(l)+\delta(l)} \, ,\\
c_2(l) &= \frac{n\lambda(l-1)
\beta(l-2)}{\left((n-1)\lambda(l)+\beta(l)+\delta(l)\right) } \nonumber\\
&\quad \times \frac{1}{\left( n\lambda(l-1)+\delta(l-1)\right)} \,
,\nonumber
\end{align}
and, the initial valuations will satisfy
\begin{align}
H(1) &=
\frac{n\lambda(0)}{(n-1)\lambda(1)+\beta(1)+\delta(1)} \\
H(2) &= \frac{1}{(n-1)\lambda(2)+\beta(2)+\delta(2)} \nonumber\\
&\quad\times\left[\frac{n(n-1)\lambda(0)\lambda(1)}{(n-1)\lambda(1)+\beta(1)+\delta(1)}+
\frac{n\lambda(1)\beta(0)}{n\lambda(1)+\delta(1)} \right]
\nonumber.
\end{align}
Clearly, the difference equations (\ref{eq:recurr_PriceDep}) are
linear homogeneous with variable coefficients
(\ref{eq:recurr_PriceDep_coeff}), and, hence, the solution for
$H(l)$ can be expressed in closed form, purely in terms of the
coefficients\cite{popenda87,mallik97,boese02}. First, define a
matrix:
\begin{equation}
\mathbf{M}_l \equiv \begin{bmatrix}
c_2(l) & c_1(l) \\
c_1(l+1)c_2(l) & c_2(l+1)+c_1(l+1)c_1(l)
\end{bmatrix}.
\end{equation}
Then, the solution sequence $\{H(l): 1\leq l \leq v-1\}$, can be
represented as a product of matrices $\{\mathbf{M}_l\}$ and the
initial valuations:
\begin{align}\label{eq:PriceDependent_Hl}
\begin{bmatrix} H(2j+1)\\
H(2j+2) \end{bmatrix} &= \mathbf{M}_{2j+1} \mathbf{M}_{2j-1}
\cdots \mathbf{M}_{3}
\begin{bmatrix}H(1)\\
H(2) \end{bmatrix} \\
&= \prod_{i=1}^{j} \mathbf{M}_{2i+1}
\begin{bmatrix}H(1)\\
H(2) \end{bmatrix} , \; 0\leq j \leq \left\lfloor\frac{v-2}{2}
\right\rfloor\nonumber .
\end{align}

Solving the above equation yields a set of two $H(l)$, one
corresponding to an odd $l$ and another to an even, for every $j$.
However, the solutions will not hold at the boundary $l=v$,
because it involves a different set of coefficients as given in
(\ref{eq:PriceDependentBidding}). Thus, the boundary solution will
be distinct and dependent on the previous two valuations:
\begin{equation}\label{eq:PriceDependent_Hl_Boundary}
\begin{split}
H(v) &= \frac{(n-1)\lambda(v-1)}{\delta(v)} H(v-1) \\
&\quad +
\frac{n\lambda(v-1)\beta(v-2)}{\delta(v)\delta(v-1)}H(v-2).
\end{split}
\end{equation}

Similarly, the solutions for $G(l)$ will be
\begin{equation} \label{eq:PriceDependent_Gl}
\begin{split}
\begin{bmatrix}  G(2j+2)\\
G(2j+3) \end{bmatrix} &= \mathbf{N}_{2j+2} \prod_{i=1}^{j}
\mathbf{M}_{2i+1}
\begin{bmatrix} H(1)\\
H(2)\end{bmatrix} , \\
&\qquad\qquad\qquad\qquad  0\leq j \leq \left\lfloor\frac{v-3}{2}
\right\rfloor ,
\end{split}
\end{equation}
where the matrix and the coefficients are
\begin{equation}
\mathbf{N}_l \equiv \begin{bmatrix} d(l) & 0 \\
0 & d(l+1)\end{bmatrix},  \textrm{ and }  d(l) =
\frac{\beta(l-1)}{n\lambda(l)+\delta(l)}\,.
\end{equation}
Again, at the boundaries $l=1$ and $l=v$, the solutions will be
different:
\begin{align}\label{eq:PriceDependent_Gl_Boundary}
G(1) &= \frac{\beta(0)}{n\lambda(1)+\delta(1)}\,,\\
G(v) &= \frac{\beta(v-1)}{\delta(v)} H(v-1) . \nonumber
\end{align}

\begin{figure}
\centering
%\includegraphics[width=3in]{ExpPayoffPerTime_PriceDependent.eps}
\caption{Payoff per unit time in the price dependent bidding
model, against nominal bid rate $\beta_0$ for various pressure
coefficients. Here $n=10$, $\lambda_0=1.0$, $r=1$, $\delta=0.5$,
and $\sigma=0$.}\label{fig:PriceDependentBid}
\end{figure}

The solutions above are general, and will hold for all price
dependent functions. Suppose now, that the dependencies are such
that $\lambda$ and $\beta$ will decrease while $\delta$ will
increase, with the price level $l$. Specifically, let a {\em
pressure coefficient} $\kappa\geq 0$ \cite{hillier64} to represent
the degree to which the attained price discourages bidding, while
$\sigma\geq 0$ represents the effect of higher prices on the
seller's tendency to sell:
\begin{align} \label{eq:PriceDependentBidding_Functions}
\beta(l) &= \frac {\beta_0}{(l+1)^{\kappa} } \,, \quad l \geq 0 , \\
\lambda(l) &= \frac{\lambda_0}{(l+1)^{\kappa}} \,, \quad l \geq 0
,\nonumber \\
\delta(l) &= l^\sigma\delta_0 , \quad l \geq 1 ,\nonumber
\end{align}
where $\beta_0$, $\lambda_0$ and $\delta_0$ are fixed nominal
rates. Although we use the same $\kappa$ for the SB and other
bidders, it is easy to relax this restriction. When $\kappa=1$, we
have the case of ``harmonic discouragement'', and if $\kappa=0$
the bidders are insensitive to price, and consequently, the whole
system reduces to the previously solved model~(\ref{eq:1}).

Now, for functionals of form
(\ref{eq:PriceDependentBidding_Functions}), the explicit solutions
for $H(l)$ will follow (\ref{eq:PriceDependent_Hl}) and
(\ref{eq:PriceDependent_Hl_Boundary}), where the coefficients
$c_1$ and $c_2$ are
\begin{align}
c_1(l) &=
\left(\frac{l+1}{l}\right)^\kappa \frac{(n-1)\lambda_0} {(n-1)\lambda_0 + \beta_0 + l^\sigma(l+1)^\kappa\delta_0}   \,,\\
c_2(l) &= \left(\frac{l+1}{l-1}\right)^\kappa
\frac{n\lambda_0\beta_0}{\left((n-1)\lambda_0+\beta_0+l^\sigma(l+1)^\kappa
\delta_0 \right) } \nonumber \\
&\quad \times \frac{1}{\left(n\lambda_0 + l^\kappa (l-1)^\sigma
\delta_0 \right)}  \,,\nonumber
\end{align}
and the initial valuations become
\begin{align}
H(1) &=
\frac{n\lambda_0 2^\kappa}{(n-1)\lambda_0+\beta_0+2^\kappa\delta_0} \,,\\
H(2) &= \frac{3^\kappa}{(n-1)\lambda_0 + \beta_0 + 2^\sigma
3^\kappa \delta_0 }\nonumber\\
&\quad \times\left[\frac{n(n-1){\lambda_0}^2}{(n-1)\lambda_0 +
\beta_0 + 2^\kappa \delta_0}+ \frac{n\lambda_0 \beta_0}
{n\lambda_0 + 2^\kappa\delta_0 } \right] \nonumber.
\end{align}
For $G(l)$, the solutions will follow the general forms
(\ref{eq:PriceDependent_Gl}) and
(\ref{eq:PriceDependent_Gl_Boundary}), where the coefficient
\begin{equation}
d(l) = \left(\frac{l+1}{l}\right)^\kappa \frac{\beta_0}{n\lambda_0
+ l^\sigma(l+1)^\kappa \delta_0} \,,
\end{equation}
and the initial valuation is $G(1) = \frac{2^\kappa \beta_0}
{n\lambda_0 + 2^\kappa\delta_0 }$.


The examples in Figure~\ref{fig:PriceDependentBid} illustrate the
effect of $\kappa$ on the expected payoff per unit time for the
SB. We see that the pressure coefficient does not make a
difference for relatively small bid rates $\beta_0$, and that a
higher coefficient fetches a better payoff rate at higher bid
rates. Also, a small increase in $\kappa$ from $0$ to $0.2$ yields
a bigger difference in payoff rates, than an equal-sized increase
from $0.8$ to $1.0$.


\section{Conclusions} \label{Conclusions}
In this paper we have considered auctions in which bidders make
offers that are sequentially increasing in value by a unit price
in order to minimally surpass the previous highest bid, and
modelled them as discrete state-space random processes in
continuous time. Analytical solutions are obtained and measures
that are of interest to the SB are derived.

The measures that can be computed in this way include the SB's
probability of winning the auction, its expected savings with
respect to the maximum sum it is willing to pay, and the average
time that the SB spends before it can make a purchase. An
extension of the model that incorporates price-dependent
behaviours of the agents has also been presented.

The model allows us to quantitatively characterise intuitive and
useful trade-offs between improving the SB's chances of buying a
good quickly, and the price that it has to pay, in the presence of
different levels of competition from the other bidders.


There are interesting extensions and applications of these models
that can be considered, such as the behaviour of bidders and
sellers that may have time constraints for making a purchase, and
the possibility of the SB's moving among different auctions so as
to optimise measures which represent its self-interest. Another
interesting area of study may be to examine bidders who are
``rich'' and are willing to drive away rivals at any cost, and who
may create different auction environments for bidders that have
significantly different levels of wealth. Yet another area of
interest concerns auctions where items are sold in batches of
varying sizes, with prices which depend on the number of items
that are being bought.

\ack{This research was undertaken as part of the ALADDIN
(Autonomous Learning Agents for Decentralised Data and Information
Networks) project and is jointly funded by a BAE Systems and EPSRC
(Engineering and Physical Research Council) strategic partnership
(EP/C548051/1).}


\nocite{*}

\bibliographystyle{compj}
\bibliography{ModellingBidders}


\end{document}
