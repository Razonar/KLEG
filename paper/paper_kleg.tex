\documentclass{comjnl}

\usepackage{amsmath}

%\copyrightyear{2014} \vol{00} \issue{0} \DOI{000}

\begin{document}


\title[Kirchhoff's laws for Linear Equations over a circuit modelled in Graphs]{Kirchhoff's laws for Linear Equations over a circuit modelled in Graphs}
\author{Victor Chaves}
\affiliation{Military Institute of Engineering, Rio de Janeiro, Brazil} \email{victor.slosilojn@gmail.com}

\shortauthors{V. Chaves}

%\received{00 January 2009}
%\revised{00 Month 2009}


%\category{C.2}{Computer Communication Networks}{Computer Networks}
%\category{C.4}{Performance of Systems}{Analytical Models}
%\category{G.3}{Stochastic Processes}{Queueing Systems}
%\terms{Internet Technologies, E-Commerce}
\keywords{Graphs, Linear Algebra, Electricity Circuits}


\begin{abstract}
Graphs and circuits are so closely related that an undergraduate course about electricity networks that does not mention it's graph modelling should be considered a scientific blasphemy. Here are presented the basic aspects of this relationship, linking the linear behaviour of generalized Ohm's Law and the linear model of Kirchoff's Law using some residual linear algebra born from graph's incidence matrix. In fact, so little graph theory is required that not knowing graph theory is a poor excuse to ignore this approach. Some simple sample circuits are analysed using a software derived directly from the model discussed and it's source code is made available so that anyone can try their own tests.
\end{abstract}

\maketitle


\section{Introduction}
\label{Sec:Intro}

By it's very definition, electrical circuits can be well modelled by graphs. The vertices may carry the nodes with voltage information if we set a ground referential and the edges may represent the electrical components, carrying their impedance.

\subsection{Electrical Circuits and Components}
\label{Sec:Circuits&Components}

% Talk about resistors, inductors and capacitors

\subsection{Kirchhoff's Laws}
\label{Sec:Kirchhoff}

% Just talk about energy conservation

\subsection{Complex Numbers and Phasor Notation}
\label{Sec:Phasors}

% Time to present complex impedance, current and voltage

\subsection{Generalized Ohm's Law}
\label{Sec:Ohm}



\section{Graph Modeling}
\label{Sec:Graph}
% Reformulate Kirchhoff's laws and Ohm's laws
Let $A$ be the circuit's incidence matrix, $\vec{i}$ the current vector, $\vec{v}$" the voltage vector and $\vec{p}$ the potential vector. The Kirchhoff Laws are now written:
\begin{itemize}
\item \emph{Node Conservation:} $A\vec{i} = \vec{0}$
\item \emph{Path Conservation:} Satisfied if by definition $A^T\vec{p}=-\vec{v}$
\end{itemize}

Being $C$ the conductance matrix of this circuit, the Ohm's Law is written as:
$$ \vec{i} = C\vec{v} $$

And hence, the 3 laws combined gives the complete linear system:
$$ ACA^T\vec{p}=\vec{0} $$

\section{Numerical Examples}


\section{Conclusions}
\label{Conclusions}


\nocite{*}

\bibliographystyle{compj}
\bibliography{ModellingBidders}


\end{document}
